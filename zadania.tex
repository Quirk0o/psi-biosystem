%%%%%%%%%%%%%%%%%%%%%%%%%%%%%%%%%%%%%%%%%%%%%%%%%%%%%%%%%%%%%%%%%%%%%%
% LaTeX Example: Project Report
%
% Source: http://www.howtotex.com
%
%%%%%%%%%%%%%%%%%%%%%%%%%%%%%%%%%%%%%%%%%%%%%%%%%%%%%%%%%%%%%%%%%%%%%%


%%% Preamble
\documentclass[paper=a4, fontsize=12pt]{scrartcl}
\usepackage[T1]{fontenc}
\usepackage{fourier}

% For encoding
\usepackage[utf8]{inputenc}
\usepackage{polski}

\usepackage[protrusion=true,expansion=true]{microtype}	
\usepackage{amsmath,amsfonts,amsthm} % Math packages

% Images and figures
\usepackage[pdftex]{graphicx}
\usepackage{epstopdf}
\usepackage{float}
\usepackage{pdfpages} % For including PDFs
\usepackage{pdflscape} % For displaying pages horizontally

\usepackage{url}
\usepackage[margin=2.5cm]{geometry}

% For better tables
\usepackage{tabularx}

\usepackage{enumitem}

%%% Custom sectioning
%\usepackage{sectsty}
%\allsectionsfont{\normalfont\scshape}


%%% Custom headers/footers (fancyhdr package)
\usepackage{fancyhdr}
\pagestyle{fancyplain}
\fancyhead{}											% No page header
\fancyfoot[L]{}											% Empty 
\fancyfoot[C]{}											% Empty
\fancyfoot[R]{\thepage}									% Pagenumbering
\renewcommand{\headrulewidth}{0pt}			% Remove header underlines
\renewcommand{\footrulewidth}{0pt}				% Remove footer underlines
\setlength{\headheight}{13.6pt}
\setlength{\parindent}{0pt}
\setlength{\parskip}{10pt}


%%% Equation and float numberingb
\numberwithin{equation}{section}		% Equationnumbering: section.eq#
\numberwithin{figure}{section}			% Figurenumbering: section.fig#
\numberwithin{table}{section}				% Tablenumbering: section.tab#

%%% Maketitle metadata
\newcommand{\horrule}[1]{\rule{\linewidth}{#1}} 	% Horizontal rule

\title{
		%\vspace{-1in} 	
		\usefont{OT1}{bch}{b}{n}
		\normalfont \normalsize \textsc{Akademia Górniczo Hutnicza} \\ [25pt]
		Wydział Informatyki, Elektroniki i Telekomunikacji
		\horrule{0.5pt} \\[1cm]
		\includegraphics[width=.35\textwidth]{img/agh_znk_wbr_cmyk.eps} \\[1.5cm]
		\huge System usprawniający deklarację i zbiórkę odpadów
		\horrule{0.5pt} \\[0cm]
}
\author{
		\normalfont \normalsize
        Mateusz Kwiecień\\[-3pt]	\normalsize
        Beata Obrok\\[-3pt]			\normalsize
        Arkadiusz Socha\\[-3pt]		\normalsize
        Dawid Suder\\[-3pt]			\normalsize
}
\date{}


%%% Begin document
\begin{document}
% \maketitle
% \tableofcontents
\clearpage

\paragraph{Beata Obrok} \ \\
	\begin{itemize}
		\item Organizacja pracy zespołu
		\item Przygotowanie środowiska i wzorów plików
		\item Kontrola czasu pracy członków zespołu 
		\item Poprawa i unifikacja diagramów
			\begin{itemize}
				\item Diagramy Przepływu Danych
				\item Diagramy Aktywności
			\end{itemize}
		\item Gromadzenie informacji w firmie
		\item Opis struktury organizacyjnej
		\item Diagram - Struktura grupy kapitałowej
		\item Diagram - Struktura organizacyjna spółki
		\item Zwięzła nazwa problemu
		\item Dokumenty wprowadzane i wyprowadzane z systemu (wzory oraz opisy)
		\item Diagramy Use Case
			\begin{itemize}
				\item Obsługa klienta
				\item Obsługa skupu
				\item Księgowość
				\item Magazyn
				\item Obsługa kierowców
			\end{itemize}
		\item Scenariusze Use Case dla Obsługi Klienta
		\item Diagram kontekstowy
		\item Diagram Przepływu Danych - Poziom 1
		\item Diagram Aktywności - Obsługa klienta
		\item Projekt interfejsu użytkownika
	\end{itemize}

\paragraph{Arkadiusz Socha} \ \\
	\begin{itemize}
		\item Obszary aktywności
		\item Opis stanowisk pracy
		\item Dane przechowywane w systemie
		\item Scenariusze Use Case: 
			\begin{itemize}
				\item Księgowość
				\item Obsługa magazynu
			\end{itemize}
		\item Diagramy Przepływu Danych
			\begin{itemize}
				\item Poziom 1 - wstępny schemta
				\item Obsługa magazynu
			\end{itemize}
		\item Roboczy słownik danych
		\item Relacje w tabeli krzyżowej
		\item Diagram ERD
		\item Diagramy aktywności
			\begin{itemize}
				\item Ogólny diagram STD
				\item Obsługa księgowości
				\item Obsługa magazynu
			\end{itemize}
		\item Równoważenie modeli
		\item Architektura systemu
	\end{itemize}

\paragraph{Dawid Suder} \ \\

\paragraph{Mateusz Kwiecień} \ \\
	\begin{itemize}
		\item Dziedzina problemu
		\item Cele przedsięwzięcia projektowego
		\item Wymagania niefunkcjonalne(z podziałem na grupy wymagań)
		\item Wymagania funkcjonalne dla dodatkowych funkcji systemu(funkcje administracyjne, wspólne i wewnętrzne)
		\item Diagramy Przepływu Danych
			\begin{itemize} 
				\item Obsługa klienta 
				\item Księgowość
			\end{itemize}
		\item Spis encji
	\end{itemize}

\end{document}