Projekt od strony teoretycznej bez implementacji, okazał się bardzo czasochłonny, jednak biorąc pod uwagę to, że w kolejnym kroku, tj. implementacji nie trzeba już myśleć nad wyglądem projektu, pozwala to na rozpisanie prac i szybsze ich zakończenie.
W ocenie zespołu, projekt spełnia założone mu wymagania i rozwijązuje postawione mu problemy. Projekt ten pokazał, jak trudnym przedsięwzięciem jest jakiegokolwiek większego projektu, w którym bierze udział wielu programistów. Największymi trudnościami okazała się bowiem komunikacja z klientem, jak i w zespole.
Dzięki temu doświadczeniu dowiedzieliśmy się, jak ważne jest utrzymanie spójnej i czytelnej dokumentacji, gdyż w kolejnych krokach użyte zostały informacje z poprzednich, najczęściej stworzonych przez kogoś innego, rozdziałach dokumentacji.