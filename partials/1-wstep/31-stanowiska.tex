
\begin{enumerate}
	\item Właściciel \\ 
	Właściciel firmy odpowiada za jej funkcjonowanie, podlegają mu pracownicy. Do jego głównych zadań należą:
		\begin{itemize}
			\item organizacja pracy podwładnych i ustalanie wynagrodzenia
			\item nadzór nad spółkami zależnymi
		\end{itemize}
	\item Pracownik działu skupu 
		\begin{itemize}
			\item pozyskiwanie oświadczeń odzysku od zewnętrznych zakładów prowadzących recykling odpadów
			\item obsługa zamówień surowców powstałych po przetworzeniu odpadów
		\end{itemize}
	\item Pracownik działu sprzedaży
		\begin{itemize}
			\item Pracownik ds. deklaracji \\
			Zajmuje się odbieraniem deklaracji o ilości odpadów, które wprowadzają do obiegu klienci firmy.
			\item Przedstawiciel handlowy\\
			Zajmuje się pozyskiwaniem nowych klientów odwiedzając placówki firm potencjalnie zainteresowanych usługami firmy.
			\item Telemarkter \\
			Zajmuje się pozyskiwaniem nowych klientów dzwoniąc do firm potencjalnie zainteresowanych usługami firmy.
		\end{itemize}
	\item Magazynier
		\begin{itemize}
			\item przygotowanie zamówień
			\item odbieranie dostaw od kierowców
			\item aktualizacja stanu magazynu
		\end{itemize}	
	\item Kierowca
		\begin{itemize}
			\item odbiór odpadów od klientów
			\item przewóz odpadów do zakładów przetwarzania odpadów
			\item dostarczanie zamówionych surowców do klientów
		\end{itemize}

	\item Księgowy
		\begin{itemize}
			\item wystawianie faktur
			\item tworzenie raportów finansowych
			\item rozliczanie firmy z Urzędem Skarbowym
		\end{itemize}
\end{enumerate}

