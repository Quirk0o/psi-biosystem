
\begin{enumerate}
	\item Obsługa klienta
		\begin{itemize}
			\item złożenie deklaracji o ilości odpadów \\
			Klient podaje informacje o rodzaju oraz ilości (lub wadze) odpadów, które wprowadził do obiegu w danym okresie rozliczeniowym poprzez formularz udostępniany przez system. Dane dotyczące deklaracji zostają zapisane w \emph{rejestrze deklaracji}.
			\item oddanie odpadów \\
			Klient za pomocą formularza udostępnianego przez system zamawia kierowcę do odbioru odpadów przeznaczonych do odsyku.
			Dane dotyczące odpadów oraz adres odbioru zostają zapisane w \emph{rejestrze ofert sprzedaży}
			\item złożenie zamówienia
			Klient za pomocą formularza udostępniane przez system wprowadza surowce, które chce zamówić, oraz adres dostawy. Dane o zamówienia zostają zapisane w \emph{rejestrze zamówień}
		\end{itemize}

	\item Obsługa skupu
		\begin{itemize}
			\item sprawdzenie stanu zamówienia \\ 
			Kupujący może w każdej chwili sprawdzić na stronie internetowej stan zamówienia, który jest udostępniany przez \emph{rejestr zamówień}.
		\end{itemize}
	\item Obsługa kierowców 
		\begin{itemize}
		\item stawienie się w magazynie po towar do wysyłki \\ 
		Kierowca stawia się w magazynie po towar, otrzymuje wcześniej informację o detalach zamówienia, pomaga magazynierowi załadadować towar.
		\item dostarczenie surówców wtórnych do placówek recyklingowych \\ 
		Kierowca dostarcza odpady, do odpowiednich placówek, pobiera fakturę od firmy zewnętrznej.
		\item odbiór odpadów
		\end{itemize}
	\item Obsługa magazynu
		\begin{itemize}
		\item aktualizacja stanu magazynu \\
	 	Magazynier uaktualnia ilość poszczególnych produktów, modyfikując rejestr stanu magazynu.
	 	\item przygotowanie towaru do sprzedaży \\
	 	Magazynier dostaje informacje o zamówieniu(poprzez rejestr zamówień), które kompletuje.
		\end{itemize}
	\item Obsługa księgowości
		\begin{itemize}
		\item Generowanie faktur \\ 
		Generowanie faktur na podstawie danych zawartych w rejestrach zamówień. System zapisuje je w rejestrze faktur.
		\item Wprowadzanie faktur \\
		Wpraowadzanie do systemu faktur, które firma otrzymuje przy zakupie oświadczeń oddzysku.
		\end{itemize}
	\item Wspomaganie pracy właściciela
		\begin{itemize}
		\item Generowanie raportów przychodów i wydatków\\
		System generuje wyżej wymienione raporty, na podstawie danych zawartych w rejestrze faktur. Raporty są zapisywane w rejestrze raportów.
		\end{itemize}

\end{enumerate}