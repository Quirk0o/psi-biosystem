
\begin{enumerate}
	\item Obsługa klienta
		\begin{itemize}
		\item złożenie deklaracji o ilości odpadów \\
		Klient podaje informacje o rodzaju oraz ilości (lub wadze) odpadów, które wprowadził do obiegu, może to zrobić poprzez stronę internetową. Informacja o ilości odpadów zostaje wpisana do \emph{rejestru deklaracji}.
		\item oddanie odpadów \\
		Klient może zamówić kierowcę, który odbierze od niego odpady przeznaczone do odzysku. Zamówienie zostaje zapisane w \emph{rejestrze ofert sprzedaży}
		\item sprawdzenie informacji o wysłaniu kierowcy \\ 
		Klient może sprawdzić na stronie internetowej, czy kierowca został już do niego wysłany. 
		\end{itemize}

	\item Obsługa skupu
		\begin{itemize}
		\item złożenie zamówienia \\
		Kupujący podaje dokładne informacje o zamówieniu, może to zrobić telefonicznie, mailowo, a także osobiście. Informacja ta zostaje zapisana do \emph{rejestru zamówień}.
		\item sprawdzenie stanu zamówienia \\ 
		Kupujący może w każdej chwili sprawdzić na stronie internetowej stan zamówienia, który jest udostępniany przez \emph{rejestr zamówień}.
		\end{itemize}
	\item Wspomaganie pracy kierowców 
		\begin{itemize}
		\item stawienie się w magazynie po towar do wysyłki \\ 
		Kierowca stawia się w magazynie po towar, otrzymuje wcześniej informację o detalach zamówienia, pomaga magazynierowi załadadować towar.
		\item dostarczenie surówców wtórnych do placówek recyklingowych \\ 
		Kierowca dostarcza odpady, do odpowiednich placówek, pobiera fakturę od firmy zewnętrznej.
		\item odbiór odpadów
		\end{itemize}
	\item Wspomaganie pracy magazynierów
		\begin{itemize}
		\item aktualizacja stanu magazynu \\
	 	Magazynier uaktualnia ilość poszczególnych produktów, modyfikując rejestr stanu magazynu.
	 	\item przygotowanie towaru do sprzedaży \\
	 	Magazynier dostaje informacje o zamówieniu(poprzez rejestr zamówień), które kompletuje.
		\end{itemize}
	\item Wspomaganie pracy księgowego
		\begin{itemize}
		\item Generowanie faktur \\ 
		Generowanie faktur na podstawie danych zawartych w rejestrach zamówień. System zapisuje je w rejestrze faktur.
		\item Wprowadzanie faktur \\
		Wpraowadzanie do systemu faktur, które firma otrzymuje przy zakupie oświadczeń oddzysku.
		\end{itemize}
	\item Wspomaganie pracy właściciela
		\begin{itemize}
		\item Generowanie raportów przychodów i wydatków\\
		System generuje wyżej wymienione raporty, na podstawie danych zawartych w rejestrze faktur. Raporty są zapisywane w rejestrze raportów.
		\end{itemize}

\end{enumerate}