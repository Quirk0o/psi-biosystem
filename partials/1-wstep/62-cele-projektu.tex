
% Cele stawiane przed zespołem projektowym, nieistotne lub drugorzędne dla zleceniodawcy albo wręcz nie ujawniane przed nim; 
% przeważnie dotyczą one sposobu prowadzenia przedsięwzięcia projektowego [np. stosowane metodyki, narzędzia] lub pobocznych, 
% niewidocznych dla użytkownika i zleceniodawcy systemu efektów procesu jego wytwarzania, np. [dodatkowy opis, wskazówki, instrukcje]

Celem przedsięwzięcia projektowego jest zapoznanie się z technikami i narzędziami wykorzystywanymi do Projektowania Systemów Informatycznych, jak i naukę pracy grupowej.

Wynikiem tych działań ma być szczegółowy projekt Systemu dla danego problemu, w naszym przypadku zaprojektowanie struktury firmy BIOSYSTEM, który pokaże nam w praktyczny sposób konieczność tych działań. Projekt jest bazowany na realnym problemie, dlatego też pomoże nam to wyrobić pewne, bardziej praktyczne niż teoretyczne, nawyki.
