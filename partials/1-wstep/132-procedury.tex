
\begin{enumerate}
	\item Obsługa sprzedaży
		\begin{itemize}
		\item złożenie deklaracji o ilośći odpadów \\
		Sprzedający podaje informacje o ilości odpadów, które ma do sprzedania, może to zrobić poprzez stronę internetową. Informacja o ilości odpadów zostaje wpisana do rejestru ofert sprzedaży.
		\item sprawdzenie informacji o wysłaniu kierowcy \\ 
		Sprzedający może sprawdzić na stronie internetowej, czy kierowca został już do niego wysłany.
		\item oddanie odpadów \\
		Sprzedający czeka na kierowcę, oddaje odpady, otrzymuje potwierdzenie odbioru.
		\end{itemize}
	\item Obsługa kupującego
		\begin{itemize}
		\item złożenie zamówienia \\
		Kupujący podaje dokładne informacje o zamówieniu, może to zrobić telefonicznie, mailowo, a także osobiście. Informacja ta zostaje zapisana do rejestru zamówień.
		\item sprawdzenie stanu zamówienia \\ 
		Kupujący może w każdej chwili sprawdzić na stronie internetowej stan zamówienia, który jest udostępniany przez rejestr zamówień.
		\item płatnośc z góry \\
		Kupujący może uiścić przedpłatę po złożeniu zamówienia.
		\item odbiór zamówienia \\
		Klient czeka na kierowcę przysłanego z firmy, odbiera towar i jeżeli nie uiścił opłaty wcześniej, robi to teraz.
		\end{itemize}
	\item Wspomaganie pracy kierowców 
		\begin{itemize}
		\item stawienie się w magazynie po towar do wysyłki \\ 
		Kierowca stawia się w magazynie po towar, otrzymuje wcześniej informację o detalach zamówienia, pomaga magazynierowi załadaować towar.
		\item stawienie się po odbiór odpadów \\
		Kierowca stawia się u sprzedającego, ma wcześniej informację o ilości surowców wtórnych(udostępione przez rejestr ofert sprzedaży), które powinien otrzymać i weryfikuje to.
		\item dostarczenie surówców wtórnych do placówek recyklingowych \\ 
		Kierowca dostarcza odpady, do odpowiednich placówek, pobiera fakturę od firmy zewnętrznej.
		\end{itemize}
	\item Wspomaganie pracy magazynierów
		\begin{itemize}
		\item aktualizacja stanu magazynu \\
	 	Magazynier uaktualnia ilość poszczególnych produktów, modyfikując rejestr stanu magazynu.
	 	\item przygotowanie towaru do sprzedaży \\
	 	Magazynier dostaje informacje o zamówieniu(poprzez rejestr zamówień), które kompletuje.
		\end{itemize}
	\item Wspomaganie pracy księgowego
		\begin{itemize}
		\item Generowanie faktur \\ 
		Generowanie faktur na podstawie danych zawartych w rejestrach zamówień. System zapisuje je w rejestrze faktur.
		\item Generowanie raportów przychodów i wydatków\\
		System generuje wyżej wymienione raporty, na podstawie danych zawartych w rejestrze faktur i w rejestrze ofert sprzedaży. Raporty są zapisywane w rejestrze raportów.
		\end{itemize}
	\item Wspomaganie pracy właściciela
		\begin{itemize}
		
		\end{itemize}

\end{enumerate}