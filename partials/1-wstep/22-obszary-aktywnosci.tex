
% Wyznaczenie obszarów aktywności, które będą omówione szczegółowo w kolejnym podrozdziale

\begin{enumerate}
	\item Obsługa sprzedającego \\
	Sprzedający może powiadomić firmę o posiadanych surowcach wtrórnych osobiście, telefonicznie, a także za pomocą strony internetowej. W zależności od ilości odpadów, a także od lokalizacji klienta, firma może wysłać kierowcę po ich odbiór, lub sprzedający może przywieźć je osobiście. Pieniądze są wpłacane na konto klienta do dwóch tygodni od odbioru towarów wtórnych.
	\item Obsługa kupującego \\
	Kupujący składa zamówienie osobiście, telefonicznie, a także za pomocą strony internetwowej. Firma wysyła kierowcę z zamówieniem, a klient odpbiera je płacąc, za dowieziony towar, chyba, że wcześniej zapłacił za niego przelewem.
	\item Wspomaganie pracy kierowców \\
	Kierowcy odbierają odpady od klientów z terenów Małopolski. Zabierają ze sobą potwierdzenie odbioru, którą wręczają sprzedającemu. Następnie przewożą odpady do:
		\begin{itemize}
			\item zewnętrznej firmy recyklingowej(opakowania z papieru, tworzyw sztucznych, szkła, blachy i aluminum),
			\item własnego zakładu przetwarzania sprzętu elektrycznego i elektronicznego.
		\end{itemize}
	Kierowcy zajmują sie także dostarczeniem zamówienia do kupującego. Zabierają ze sobą fakturę, którą wręczają kupującemu.
	\item Wspomaganie pracy magazynierów \\
	Pracownicy magazynu grupują produkty otrzymane po recyklingu. Zajmują się także przygotowaniem 
	\item Wspomaganie pracy pracownika działu sprzedaży \\
	Dział sprzedaży zajmuje się pozyskiwaniem klientów, którzy kupią odpady poddane już recyklingowi
\end{enumerate}
