
% Wyznaczenie obszarów aktywności, które będą omówione szczegółowo w kolejnym podrozdziale

\begin{enumerate}
	\item Obsługa klienta \\
		\begin{itemize}
			\item pozyskiwanie nowych klientów poprzez marketing i telemarketing
			\item obsługa deklaracji recyklingowych klientów wprowadzanych przez aplikację internetową lub doręczanych mailowo: 
				\begin{itemize}
					\item sprawdzanie ich poprawności
					\item wprowadzanie do systemu
					\item zawieranie aneksów do umowy z klientem
				\end{itemize}
			\item sprzedaż odzyskanych materiałów w \emph{Zakładzie Przetwarzania}
		\end{itemize}

	\item Obsługa skupu \\
		\begin{itemize}
			\item organizacja zbiórek odpadów oraz przeworzenie ich do \emph{Zakładu Przetwarzania BIOSYSTEM} lub do zewnętrznych zakładów przetwarzania
			\item zakup oświadczeń odzysku od zewnętrznych zakładów przetwarzania
		\end{itemize}

	\item Obsługa kierowców \\
	Kierowcy odbierają odpady od klientów z terenów Małopolski. Zabierają ze sobą potwierdzenie odbioru, którą wręczają sprzedającemu. Następnie przewożą odpady do:
		\begin{itemize}
			\item zewnętrznej firmy recyklingowej(opakowania z papieru, tworzyw sztucznych, szkła, blachy i aluminum),
			\item własnego zakładu przetwarzania sprzętu elektrycznego i elektronicznego.
		\end{itemize}
	Kierowcy zajmują sie także dostarczeniem zamówienia do kupującego. Zabierają ze sobą fakturę, którą wręczają kupującemu.

	\item Obsługa magazynu \\
	Pracownicy magazynu grupują produkty otrzymane po recyklingu. Zajmują się także przygotowaniem zamówień.

	\item Obsługa księgowości \\
	Księgowi przygotowują faktury dla klientów oraz zajmują się rozliczaniem z Urzędem Skarbowym.
\end{enumerate}
