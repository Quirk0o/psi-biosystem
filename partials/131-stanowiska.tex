
\paragraph{Opis stanowisk pracy} \ \\
\begin{itemize}
	\item Właściciel \\ 
	Właściciel firmy odpowiada za jej funkcjonowanie, podlegają mu pracownicy. Do jego głównych zadań należą:
	\begin{itemize}
		\item organizacja pracy podwładnych i ustalanie wynagrodzenia
		\item nadzór nad spółkami zależnymi
	\end{itemize}
	\item Pracownik działu skupu \\
	Głównym jego zadaniem jest odebranie deklaracji o ilości odpoadów od sprzedającego, przygotowanie potwierdzenia odbioru, oraz przekazanie go kierowcy.
	\item Pracownicy działu sprzedaży
	\begin{itemize}
	\item Przedstawiciel handlowy \\
	Zajmuje się pozyskiwaniem nowych klientów odwiedzając placówki firm potencjalnie zainteresowanych usługami naszej firmy.
	\item Telemarkter \\
	Zajmuje się pozyskiwaniem nowych klientów dzwoniąc do firm potencjalnie zainteresowanych usługami naszej firmy.
	\end{itemize}
	\item Magazynier \\
	Do jego głównych zadań należą:
		\begin{itemize}
		\item oprzygotowanie zamówień
		\item odbieranie dostaw od kierowców
		\item aktualizacja stanu magazynu
		\end{itemize}
	\item Kierowca \\
	Odbiera odpady od sprzedających, przewozi je do miejsc, gdzie są poddawane recyklingowi. Przewozi także produkty recyklingu do magazynu, a także zawozi je do kupującego.
	\item Księgowy \\
	Zajmuje się wystawianiem faktur, tworzeniem raportów finansowych oraz rozliczenianiem firmy z urzędem skarbowym.
	
\end{itemize}

