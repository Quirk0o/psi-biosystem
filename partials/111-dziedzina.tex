
% Firma, cel, dziedzina, schemat struktury organizacyjnej, słownik pojęć biznesowych (lub odsyłacz do załącznika)
% Ogólne omówienie zakresu i charakteru działalności jednostki dla której przeznaczony jest produkt tj. realizowany [projektowany] system).
% Dołączyć i opisać schemat struktury organizacyjnej jednostki. 

\paragraph{Wstęp}
\textbf{BIOSYSTEM S.A.} działa jako spółka matka konsolidująca grupę firm działających w sektorze ochrony środowiska i gospodarki odpadami. Spółka pełni funkcję dostarczyciela kapitału, inkubatora projektów w ramach grupy oraz zapewnia współdziałanie poszczególnych podmiotów funkcjonujących na równoległych rynkach.

\paragraph{Zakres i charakter działalności spółki}

Pierwszym przedmiotem działalności spółki jest nadzór właścicielski nad dwoma organizacjami odzysku wspierającymi przedsiębiorców w zakresie realizacji ustawowych obowiązków zbiórki, recyklingu i odzysku odpadów opakowaniowych, zużytych baterii oraz sprzętu elektrycznego i elektronicznego.
Kolejnym, bardzo ważnym, przedmiotem działalności BIOSYSTEM S.A. jest zbiórka, przetwarzanie i unieszkodliwianie zużytego sprzętu elektrycznego i elektronicznego (ZSEiE). Spółka ta prowadzi również najnowocześniejszy w Polsce Zakład Przetwarzania ZSEiE ze specjalną linią do utylizacji urządzeń chłodniczych. Rezultatem tego jest sprzedaż produktów i surowców powstałych w wyniku przetworzenia.
BIOSYSTEM S.A. zajmuje się również importem urządzeń elektrycznych i elektronicznych. W związku z tym prowadzi także sprzedaż detaliczną i hurtową takiego sprzętu.
Ostatnim przedmiotem działalności jest organizacja publicznych kampanii edukacyjnych wykonywanych na zlecenie spółek zależnych.