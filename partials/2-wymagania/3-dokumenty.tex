% rodzaje, wzory dokumentów zebranych lub „odtworzonych”

\subsubsection{Dokumenty wprowadzane}

	\paragraph{Dane klienta} $\rightarrow$ Projekt interfejsu 1.1. \\
	Ekstrakt z umowy, która jest dokumentem zewnętrznym systemu (nie jest przechowywana w formie elektronicznej).
	Umowa jest zawarta pomiędzy klientem a firmą BIOSYSTEM i świadczy o przejęciu od klienta obowiązku odzysku odpadów przez firmę BIOSYSTEM.

	Zawiera istotne dane klienta (informacje dot. firmy, zakres świadczonych usług (rodzaje odpadów, okresy rozliczeniowe) oraz dane użytkownika w systemie (nazwa użytkownika i hasło klienta).

	\paragraph{Deklaracja} $\rightarrow$ Załącznik 1.2 \\
	Dokument składany przez \emph{Klienta z umową} dotyczący wprowadzonych do obiegu opakowań, baterii, sprzętu elektrycznego i elektronicznego oraz ich ilości i/lub wagi.

	\paragraph{Oferta sprzedaży} $\rightarrow$ Projekt interfejsu 1.2.\ \\
	Dokument składany w systemie przez osobę prywatną lub firmę dotyczący posiadanych przez klienta odpadów przeznaczonych do odbioru przez firmę BIOSYSTEM oraz późniejszej utylizacji w \emph{Zakładzie Przetwarzania Odpadów}.

	\paragraph{Zlecenie odbioru} \ \\
	Zlecenie generowane na podstawie \emph{Oferty sprzedaży} zawierające adres klienta, który zlecił odbiór odpadów oraz adres placówki do której zawiezione zostaną odpady w celu ich recyklingu.

	\paragraph{Zamówienie} \ \\
	Zamówienie składane przez klienta na surowce pochodzące z \emph{Zakładu Przetwarzania Zużytego Sprzętu Eletrycznego i Elektronicznego BIOSYSTEM SA}.

	\paragraph{Oświadczenie odzysku} \ \\
	Dokument kupowany od \emph{Zewnętrznych Zakładów Przetwarzających Odpady}, który opisuje rodzaje odzyskanych odpadów oraz ich ilości.

	Po przejęciu obowiązku odzysku odpadów od klienta firma BIOSYSTEM zleca \emph{Zewnętrznym Zakładom Przetwarzającym Opady} odzyskanie ilości odpadów odpowiadającym wprowadzonym do obiegu przez klienta.

\subsubsection{Dokumenty wyprowadzane}
	\paragraph{Faktura VAT} $\rightarrow$ Załącznik 1.3 \\
	Faktura wystawiana dla klienta za usługę przejęcia obowiązku recyklingu deklarowanych odpadów lub za zamówione surowce.

	\paragraph{Potwierdzenie odbioru} \ \\
	Dokument podpisywany przez klienta poświadczający odbiór przez kierowcę odpadów przeznaczonych do odzysku.
