% rodzaje, wzory dokumentów zebranych lub „odtworzonych”

\subsubsection{Dokumenty wprowadzane}

	\paragraph{Umowa} \ \\
	Dokument zawarty pomiędzy klientem a firmą BIOSYSTEM świadczący o przejęciu od klienta obowiązku odzysku odpadów przez firmę BIOSYSTEM.

	Zawiera on dane klienta, zakres świadczonych usług (rodzaje odpadów, okresy rozliczeniowe) oraz dane użytkownika w systemie (nazwa użytkownika i hasło klienta).

	\paragraph{Deklaracja} $\rightarrow$ Załącznik 1.1 \\
	Dokument składany przez \emph{Klienta z umową} dotyczący wprowadzonych do obiegu opakowań, baterii, sprzętu elektrycznego i elektronicznego oraz ich ilości i/lub wagi.

	Na podstawie deklaracji wykupowane jest \emph{Oświadczenie odzysku}.

	\paragraph{Oferta sprzedaży} \ \\
	Dokument składany w systemie przez osobę prywatną lub firmę dotyczący posiadanych przez klienta odpadów przeznaczonych do odbioru przez firmę BIOSYSTEM oraz późniejszej utylizacji w \emph{Zakładzie Przetwarzania Odpadów}.

	\paragraph{Zamówienie} \ \\
	Zamówienie składane przez klienta na surowce pochodzące z \emph{Zakładu Przetwarzania Zużytego Sprzętu Eletrycznego i Elektronicznego BIOSYSTEM SA}.

	\paragraph{Oświadczenie odzysku} \ \\
	Dokument kupowany od \emph{Zewnętrznych Zakładów Przetwarzających Odpady}, który opisuje rodzaje odzyskanych odpadów oraz ich ilości.

	Po przejęciu obowiązku odzysku odpadów od klienta firma BIOSYSTEM zleca \emph{Zewnętrznym Zakładom Przetwarzającym Opady} odzyskanie ilości odpadów odpowiadającym wprowadzonym do obiegu przez klienta.

\subsubsection{Dokumenty wyprowadzane}
	\paragraph{Faktura VAT} $\rightarrow$ Załącznik 1.2 \\
	Faktura wystawiana dla klienta za usługę przejęcia obowiązku recyklingu deklarowanych odpadów lub za zamówione surowce.

	\paragraph{Potwierdzenie odbioru} \ \\
	Dokument podpisywany przez klienta poświadczający odbiór przez kierowcę odpadów przeznaczonych do odzysku.
