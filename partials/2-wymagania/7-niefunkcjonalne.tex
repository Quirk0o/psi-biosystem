
\begin{enumerate}
\item Użyteczność
	\begin{itemize}
		\item interakcja z użytkownikiem powinna być czytelna
		\item użycie prostych, estetycznych i czytelnych czcionek
		\item użycie intuicyjnych kolorów, np. dla błędnych danych koloru czerwonego
		\item w miarę możliwości składanie deklaracji powinno być analogiczne do wypełniania innych formularzy internetowych
		\item bez zbyt złożonych efektów wizualnych, które będą widocznie spowalniały działanie systemu na słabszym sprzęcie
	\end{itemize}

\item Niezawodność
	\begin{itemize}
		\item czas awarii nie może być dłuższy niż dwie godziny
		\item każdy użytkownik ma dostęp do systemu z określonymi, niezbędnymi uprawnieniami
	\end{itemize}

\item Wydajność
	\begin{itemize}
		\item system będzie w stanie dodać nową deklarację w mniej niż 5s
		\item system będzie w stanie wygenerować nową fakturę w mniej niż 5s
		\item system wyeksportuje dane archiwalne o wielkości do 100 mb w mniej niż 10s
	\end{itemize}

\item Wspieralność
	\begin{itemize}
		\item system powinien być skalowalny, tj. dodanie nowej funkcji przetwarzania/przedstawienia danych podanych przez klienta nie powinno powodować zmiany wcześniej napisanego kodu
		\item system napisany będzie w języku Java
		\item strona internetowa będzie używała języka JavaScript
	\end{itemize}

\item Harmonogram
	\begin{itemize}
		\item I miesiąc - analiza wymagań
		\item II, III miesiąc - projekt wstępny
		\item IV - VII miesiąc - projekt szczegółowy
		\item VIII - XII miesiąc - implementacja
	\end{itemize}


\end{enumerate}