% zebrane wymagania niefunkcjonalne; np. co do własności produktu,
% obowiązkowego czasu przechowywania danych archiwalnych, okresowości
% przygotowywania raportów, współpracy z innymi systemami, itp.
% zaklasyfikować do głównych grup: [produktowe, organizacyjne,
% zewnętrzne]

1) Bezpieczeństwo
	W systemie przechowywane są wszystkie dane dotyczące firm współpracujących i pracowników, dlatego też dane powinny być chronione przed nieupoważnionym dostępem.
	Utrata danych, bądź ich kradzież może spowodować duże straty finansowe firmy.
	Dane powinny być często backup'owane.

2) Stabilność
	System powinien być dostępny cały czas, a w sytuacji pesymistycznej przynajmniej w czasie pracy biura.
	Brak możliwości korzystania z systemu powoduje opóźnienia w realizacji zamówień i niemożność ich złożenia.

3) Łatwość nauki i obsługi
	System musi być na tyle intuicyjne i łatwy w obsłudze, aby nie trzeba było przeznaczać pieniędzy na doszkalanie nowych pracowników.
	Powinien być także funkcjonalny, aby maksymalnie zaoszczędzić czas na najczęściej wykonywane czynności.
