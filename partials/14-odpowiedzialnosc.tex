
% Szczegółowe wyznaczenie, jaka część obszaru modelowania będzie [lub jest] objęta funkcjami realizowanymi przez opracowywany system).
% Model (opisowy) stanu istniejącego powinien uwzględniać charakterystykę wyodrębnionych „obszarów aktywności” systemu
% (podsystemów) oraz bardzo szczegółowy opis procedur biznesowych. Stanowią one podstawę do konstrukcji (lub są nimi) biznesowych
% przypadków użycia, a następnie systemowych przypadków użycia. Uzupełnić słownikiem pojęć biznesowych 
% (zamieszczonym w Dodatku, referencje w tekście) 

W zakres odpowiedzialności systemu wchodzą wszystkie procedury z pkt 1.3.
Główną odpowiedzialnością naszego systemu jest zarządzanie strukturą firmy, pomoc 
w zautomatyzowaniu niektórych aktywności, które teraz wykonywane są ręcznie np. przesyłanie dokumentów 
pomiędzy sektorami firmy (zamówienie -> kierowca), kontrola stanu magazynu, kontrola stanu "przesyłki".
