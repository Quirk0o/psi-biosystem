\begin{figure}[H]
	\centering
	\caption{Opis procedur biznesowych}
\end{figure}

\paragraph{Opis procedur biznesowych} \ \\
\begin{enumerate}
	\item Obsługa sprzedającego
		\begin{itemize}
		\item złożenie deklaracji o ilośći odpadów \\
		\paragraph Sprzedający podaje informacje o ilości odpadów, które ma do sprzedania, może to zrobić poprzez stronę internetową.
		\item sprawdzenie informacji o wysłaniu kierowcy \\
		\paragraph  Sprzedający może sprawdzić na stronie internetowej, czy kierowca został już do niego wysłany.
		\item oddanie odpadów \\
		\paragraph Sprzedający czeka na kierowcę, oddaje odpady, otrzymuje potwierdzenie odbioru.
		\end{itemize}
	\item Obsługa kupującego
		\begin{itemize}
		\item złożenie zamówienia \\
		\paragraph Kupujący podaje dokładne informacje o zamówieniu, może to zrobić telefonicznie, mailowo, a także osobiście.
		\item sprawdzenie stanu zamówienia \\
		\paragraph Kupujący może w każdej chwili sprawdzić na stronie internetowej stan zamówienia.
		\item płatnośc z góry \\
		\paragraph Kupujący może uiścić przedpłatę po złożeniu zamówienia.
		\item odbiór zamówienia \\
		\paragraph Klient czeka na kierowcę przysłanego z firmy, odbiera towar i jeżeli nie uiścił opłaty wcześniej, robi to teraz.
		\end{itemize}
	\item Wspomaganie pracy kierowców
		\begin{itemize}
		\item stawienie się w magazynie po towar do wysyłki \\
		\paragraph Kierowca stawia się w magazynie po towar, otrzymuje wcześniej informację o detalach zamówienia, pomaga magazynierowi załadaować towar.
		\item stawienie się po odbiór odpadów \\
		\paragraph Kierowca stawia się u sprzedającego, ma wcześniej informację o ilości surowców wtórnych, które powinien otrzymać i weryfikuje to.
		\item dostarczenie surówców wtórnych do placówek recyklingowych \\
		\paragraph Kierowca dostarcza odpady, do odpowiednich placówek, pobiera fakturę od firmy zewnętrznej.
		\end{itemize}
	\item Wspomaganie pracy magazynierów
		\begin{itemize}
		\item aktualizacja stanu magazynu \\
	 	\paragraph Magazynier uaktualnia ilość poszczególnych produktów.
	 	\item przygotowanie towaru do sprzedaży \\
	 	\paragraph Magazynier dostaje informacje o zamówieniu, które kompletuje.
		\end{itemize}
\end{enumerate}