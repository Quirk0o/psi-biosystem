%%%%%%%%%%%%%%%%%%%%%%%%%%%%%%%%%%%%%%%%%%%%%%%%%%%%%%%%%%%%%%%%%%%%%%
% LaTeX Example: Project Report
%
% Source: http://www.howtotex.com
%
%%%%%%%%%%%%%%%%%%%%%%%%%%%%%%%%%%%%%%%%%%%%%%%%%%%%%%%%%%%%%%%%%%%%%%


%%% Preamble
\documentclass[paper=a4, fontsize=11pt]{scrartcl}
\usepackage[T1]{fontenc}
\usepackage{fourier}

\usepackage{polski}
\usepackage[utf8]{inputenc}
\usepackage[protrusion=true,expansion=true]{microtype}	
\usepackage{amsmath,amsfonts,amsthm} % Math packages
\usepackage[pdftex]{graphicx}	
\usepackage{url}


%%% Custom sectioning
\usepackage{sectsty}
\allsectionsfont{\normalfont\scshape}


%%% Custom headers/footers (fancyhdr package)
\usepackage{fancyhdr}
\pagestyle{fancyplain}
\fancyhead{}											% No page header
\fancyfoot[L]{}											% Empty 
\fancyfoot[C]{}											% Empty
\fancyfoot[R]{\thepage}									% Pagenumbering
\renewcommand{\headrulewidth}{0pt}			% Remove header underlines
\renewcommand{\footrulewidth}{0pt}				% Remove footer underlines
\setlength{\headheight}{13.6pt}


%%% Equation and float numbering
\numberwithin{equation}{section}		% Equationnumbering: section.eq#
\numberwithin{figure}{section}			% Figurenumbering: section.fig#
\numberwithin{table}{section}				% Tablenumbering: section.tab#


%%% Maketitle metadata
\newcommand{\horrule}[1]{\rule{\linewidth}{#1}} 	% Horizontal rule

\title{
		%\vspace{-1in} 	
		\usefont{OT1}{bch}{b}{n}
		\normalfont \normalsize \textsc{School of random department names} \\ [25pt]
		\horrule{0.5pt} \\[0.4cm]
		\huge This is the title of the template report \\
		\horrule{2pt} \\[0.5cm]
}
\author{
		\normalfont 								\normalsize
        Firstname Lastname\\[-3pt]		\normalsize
        \today
}
\date{}


%%% Begin document
\begin{document}
\maketitle
\tableofcontents

\section{Sformułowanie zadania projektowego}

	\subsection{Obszar i przedmiot modelowania}

		\subsubsection{Dziedzina problemu}

	\subsection{Obszar modelowania}

		\subsubsection{Opis struktury organizacyjnej}

		\subsubsection{Obszary aktywności}

	\subsection{Opis obszarów aktywności}

		\subsubsection{Opis stanowisk pracy}

		\subsubsection{Opis procedur biznesowych}

	\subsection{Zakres odpowiedzialności systemu}

	\subsection{Nazwa problemu}

	\subsection{Cele do osiągnięcia}

		\subsubsection{Cele produktu}

		\subsubsection{Cele przedsięwzięcia projektowego}

% \section{Opis wymagań}

% 	\subsection{Funkcje systemu}

% 	\subsection{Dane przechowywane w systemie}

% 	\subsection{Dokumenty wprowadzane i wyprowadzane z systemu}

% 	\subsection{Wymagania specjalne i ograniczenia}

% 	\subsection{Analiza wymagań funkcjonalnych}

% 	\subsection{Wymagania funkcjonalne dla dodatkowych funkcji systemu}

% 	\subsection{Wymagania niefunkcjonalne}

% \section{Analiza funkcjonalna systemu}

% 	\subsection{Diagram kontekstowy}

% 	\subsection{Analiza top-down}

% 	\subsection{Opis procesów}

% \section{Roboczy słownik danych}

% \section{Analiza struktur danych w przechowywanych magazynach}

% \section{Obraz zachowania systemu w czasie}

% \section{Równoważenie modeli}

% \section{Architektura systemu}

% 	\subsection{Architektura całego systemu}

% 	\subsection{Architektura podsystemów}

% 	\subsection{Wewnętrzna architektura podsystemów}

% \section{Projekt interfejsu użytkownika}

% \section{Podsumowanie}

% 	\subsection{Założenia implementacyjne}

% 	\subsection{Weryfikacja projektu systemu}

% 	\subsection{Uwagi i wnioski końcowe}

%%% End document
\end{document}